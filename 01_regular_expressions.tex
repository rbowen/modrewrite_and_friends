\part{Regular Expressions}

\chapter{Introduction to Regular Expressions}
\index{Regular Expressions}
\label{chapter_regex}

Much of the content in this book requires that you have some mastery
of regular expressions. Indeed, in my years of teaching mod\_rewrite,
it has been my observation that people don't find mod\_rewrite hard at
all: they're just intimidated by regular expressions.

There is one excellent book about regular expressions, and if you want
to become a regular expression (or "regex") guru, you should get it. That
book is \textbf{Mastering Regular Expressions} (http://regex.info/book.html) by
Jeffrey Friedl.

If you want to know enough about regex to master mod\_rewrite, read
this chapter a few times, and that should be sufficient.

The goal of this chapter is to introduce the building blocks - the basic vocabulary - and 
then discuss some of the arcana of crafting your own regular expressions, as well as reading 
those that others have bequeathed to you.

If you are already reasonably familiar with regex syntax, you can safely skip 
this chapter.

\section{The Building Blocks}
\index{Regex building blocks}

Regular expressions are a means to describe a text pattern (technically, it's any data, but 
in the context of Apache httpd, we're primarily interested in text as it appears in URLs), so that you can look for that pattern in a block of data. The 
best way to read any regular expression is one character at a time, so you need to know what 
each character represents.

These are the basic building blocks that you will use when writing regular expressions. If 
you don't already know regex syntax, you'll want to stick a bookmark on this page, since you'll be 
referring to it until you become familiar with these characters. Table \ref{regex_vocab_table} is your key to 
translating a line of seemingly random characters into a meaningful pattern. The table will be 
followed by further explanations and examples for each of the items in the table.

\begin{table}[ht]
\caption{Regular expression vocabulary}
\label{regex_vocab_table}
\begin{tabular}{l | p{12cm}}
Character &  Meaning\\ \hline \hline
\verb=.= &  Any character\\ \hline
\verb=\= & Escapes a character that has a special meaning. Thus, \verb=\.=
means a literal \verb=.= character. You can match a literal \verb=\=
character by using \verb=\\=. Additionally, placing \verb=\= 
in front of a regular 
character can add a special meaning to that character. For example,
\verb=\t= means a tab character.\\ \hline
\verb=^= & An anchor which insists that the pattern start at the beginning of the 
string. \verb=^A= means that the string must start with A.\\ \hline
\verb=$= & An anchor which insists that the string ends with the specified 
pattern. \verb=X$= means that the string must end with X.\\ \hline
\verb=+= &
Match the previous thing one or more times. So \verb=a+= means "one or 
more a's."\\ \hline
\verb=*= &
Match the previous thing zero or more times. This is the same as \verb=+=, 
except that it's also acceptable if the thing wasn't there at all.\\
\hline
\verb=?= &
Match the previous thing zero or one times. In other words, it makes it 
optional. It also makes the \verb=*= and \verb=+= characters non-greedy.\footnote{See section below on Greedy Matching - \ref{greedy}}\\ \hline
\verb={n,m}= &
Indicates that the previous thing should match at least n, and not more
than m times. For example, \verb=a{2,7}= matches at least 2, and not
more than 7, occurrences of the letter a. \\ \hline
\verb=( )= & 
Provides grouping and capturing functions. Grouping means treating 
more than one character as though they were a single unit. You can apply
repetition characters to a group created in this way.

Capturing means remembering the thing that matched, so that we can use it 
again later. This is called a 'backreference.'
\\ \hline
\verb=[ ]= &
Called a "character class," this matches only one of the contained 
characters. For example, \verb=[abc]= matches a single character which is 
either a or b or c.\\ \hline
\verb=^= &
Negates a match within a character set. Careful. This appears to be a 
contradiction, but it's not. The \verb=^= character, unfortunately, means 
different things in different contexts. Thus, \verb=[^abc]= matches a single 
character which is neither a nor b nor c.\\ \hline
\verb=!= &
Placed on the front of a regular expression, this means "NOT". That 
is, it negates the match, and so succeeds only if the string does not 
match the pattern.\\ \hline
\end{tabular}
\end{table}

That's not all there is to regular expressions, but it's a really good starting point. 
Each regular expression presented in this book will have an explanation of what it's doing, 
which will help you see in practical examples what each of the above characters actually ends 
up meaning in the wild. And, in my experience, regular expressions are understood much 
more quickly via examples rather than via lectures.

What follows is a more detailed explanation of each of the items in the table above, with 
examples.

\subsection{Matching anything}
\index{Wildcard}
\index{.}

The \verb=.= character in a regular expression matches any character. For example, consider the 
following pattern:

\verb=a.c=

That pattern matches a string containing a, followed by any character, followed by c. So, 
that pattern matches the strings "abc", "ancient", and "warcraft", each of which contain 
that pattern. It does not match "tragic", on the other hand, because there are two characters 
between the a and the c. That is, the \verb=.=, by itself, matches a single character only.

The \verb=.= character is very frequently used in connection with
\verb=*= to mean ``match everything''. You'll see the \verb=(.*)=
pattern appearing often throughout this book, and throughout examples
that you see online. And while it's often what you want, it's just as
often used incorrectly. Remember that while \verb=(.*)= matches any
string, so will the simpler and faster pattern \verb=^= because every
string has a start (even an empty string) and so \verb=^= matches it.

It's faster because while \verb=(.*)= has to match all the way out to
the end of the string, \verb=^= only has to note that the string has a
beginning, and then it is done. Note also that the pattern \verb=(.*)=
has parenthesis and therefore captures the matched string into the
variable \verb=$1=. If you're not planning to use \verb=$1= in a later
substitution, then this, in addition to being a waste of computation
cycles, is a waste of memory.

While considerations of this kind probably won't save you a noticeable
amount of time, getting into the habit of writing efficient regular
expressions will, in the long run, not only save you these small
amounts, but will result in rules that are easier to understand and
easier to maintain, because they match only what you're interested in,
and nothing more.

\subsection{Escaping characters}
\index{Escape characters}
\index{Metacharacters}

The backslash, or escape character, either adds special meaning to a character, or removes it, 
depending on the context. For example, you've already been told that the \verb=.= character has 
special meaning. But if you want to match the literal \verb=.= character, then you need to escape it 
with the backslash. So, while \verb=.= means "any character," \verb=\.= means a literal "." character.

Conversely, some characters gain special meaning when prefixed by a \ character. For 
example, while s means a literal "s" character, \verb=\s= means a "whitespace" character. That is, a 
space or a tab.

Table \ref{metachar_table} lists useful escape characters that you'll
see throughout the book and can be used as shorthand for more
verbose patterns.

\begin{table}[ht]
\caption{Useful escape characters}
\label{metachar_table}
\begin{tabular}{l | p{12cm}}
Character & Meaning\\ \hline \hline
\verb#\d# & Match any character in the range 0 - 9\\ \hline
\verb#\D# & Match any character NOT in the range 0 - 9\\ \hline
\verb#\s# & Match any whitespace characters (space, tab etc.).\\ \hline
\verb#\S# & Match any character NOT whitespace (space, tab).\\ \hline
\verb#\w# & Match any character in the range 0 - 9, A - Z and a - z\\ \hline
\verb#\W# & Match any character NOT the range 0 - 9, A - Z and a - z\\ \hline
\verb#\b# & Word boundary. Match any character(s) at the beginning (\verb#\babc#) and/or
end (\verb#abc\b#) of a word, thus \verb#\bcow\b# will find cow but not cows, but \verb#\bcow#
will find cows. \\ \hline
\verb#\B# & Not a word boundary. Match any character(s) NOT at the beginning(\verb#\Babc#)
and/or end (\verb#xx\B#) of a word, thus \verb#\Bcow\B# will find scows but not
cows, but \verb#cow\B# will find coward.\\ \hline
\verb=\t= & Match a tab character \\ \hline
\verb=\n= & Match a newline character \\ \hline
\verb=\x= & Matches a character with a particular hex code. For example,
\verb=\x5A= would match a Z, which has a hex code of 5A.\\ \hline
\end{tabular}
\end{table}

The term "metacharacter" is often also applied to characters such as \verb=.= and \verb=$= which have special 
meanings within regular expressions.

\subsection{Anchoring text}
\index{Anchors}
\index{\verb=^=}
\index{\verb=$=}

Referred to as anchor characters, these ensure that a string starts with, or ends with, a 
particular character, or sequence of characters. Since this is a very common need, these are 
included in this basic vocabulary. Consider the examples in table \ref{anchorexamples}:

\begin{table}[ht]
\caption{Examples of using pattern anchors}
\label{anchorexamples}
\begin{tabular}{l | p{12cm}}
Example & Meaning\\ \hline \hline
\verb=^/= & This matches any string that starts with a slash \\ \hline
\verb=\.jpg$= & This pattern matches any string that ends with .jpg. \\ \hline
\verb=^/$= & Matches a string that starts with, and ends with, a slash. That is, it will only 
match a string that is a single slash, and nothing else. \\ \hline
\verb=^$= & Matched an empty string - that
is, a string that has nothing between its start and its end. \\ \hline
\end{tabular}
\end{table}

Remember, as you craft your regular expressions, that they are, by
default, a substring match. Which is to say, a pattern of \verb=cow=
matches cow, scow, coward, and pericowperitis, because they all
contain ``cow'' somewhere in them. Using the anchor characters allow you
to be more specific as to what you wanted to match. The \verb=\b=
metacharacter, introduced above, can also be useful in some contexts,
but perhaps less so when you're dealing with URLs.

\subsection{Matching one or more characters}
\index{+}

The + character allows a pattern or character to match more than once. For example, the 
following pattern will allow for common misspellings of the word "giraffe".

\verb=giraf+e+=

This pattern will allow one or more f's, as well as one or more e's. So it matches 
``girafe'', ``giraffe'', and ``giraffee''. It will also match ``girafffffeeeeee''.

Be sure to use \verb=+= rather than \verb=*= when you want to ensure
non-empty matches.

\subsection{Matching zero or more characters}
\index{*}

The * character allows the previous character to match zero or more times. That is to say, it's
exactly the same as +, except that it also allows for the pattern to not match at all. This is
often used when + was meant, which can result in some confusion when it matches an empty
string. As an example, we'll use a slight modification of the pattern used in the above
section:

\verb=giraf*e*=

This pattern matches the same strings listed above (``giraffe'', ``girafe'' and ``giraffee'')
but will also match the string "giraeeeee", which contains zero "f" characters, as well as the
string "gira", which contains zero "f" characters and zero "e" characters.

Most commonly, you'll see it used in conjunction with the . character, meaning ``match
anything.'' Frequently, in that case, the person using it has forgotten that regular expressions
are substring matches. For example, consider this pattern:

\verb=.*\.gif$=

The intent of that pattern is to match any string ending in .gif. The \verb=$= insists that it is at the 
end of the string, and the \verb=\= before the . makes that a literal . character, rather than the wildcard 
. character. In this particular case, the \verb=.*= was there to mean "starts with anything," but is 
completely unnecessary, and will only serve to consume time in the matching process.

A more useful example of the \verb=*= character is one which checks for a comment line in an 
Apache configuration file. The first non-space character needs to be a \verb=#=, but the spaces are 
optional:

\verb=^\s*#=

This pattern, then, matches a string that might (but doesn't have to) begin with 
whitespace, followed by a \verb=#=. This ensures that the first non-space character of the line is a \verb=#=.

\subsection{Repetition quantifiers}
\index{\verb={n,m}=}

If you want to match a particular number of times, you can use the
\verb={n,m}= quantifier to specify the range of times you wish to match.
The possibilities of how you can specify this are shown in table
\ref{repetition_quantifiers}.

\begin{table}[ht]
\caption{Repetition quantifiers}
\label{repetition_quantifiers}
\begin{tabular}{l | p{12cm}}
Pattern & Meaning\\ \hline \hline
\verb={n}= & Match exactly n times \\ \hline
\verb={n,}= & Match at least n times \\ \hline
\verb={n,m}= & Match at least n times, but not more than m times \\ \hline 
\end{tabular}
\end{table}

\subsection{Greedy Matching}
\index{Greedy matching}
\label{greedy}

In the case of all of the repetition characters above, matching is greedy. That is, the regular 
expression matches as much as it possibly can. That is, if you apply the regular expression 
\verb=a+= to the string \verb~aaaa~, matches the entire string, and not be satisfied by just the first 
a. This is particularly important when you are using the \verb=.*= syntax, which can 
occasionally match more than you thought it would. I'll give some examples of this after 
we've discussed a few more metacharacters.

On the other hand, if you wish for matches to not be greedy, you can
offset the greedy nature of the repetition character by putting a ?
after it.

Consider, for example, a scenario where I want to match everything between two
slashes in a URL. I'll be applying the regular expression to the URI
\verb=/one/two/three/=, and I'll try a greedy, and not-greedy, regular
expression. Table \ref{table_greedy_example} shows the results of these
patterns.

\begin{table}[ht]
\caption{Greedy vs non-greedy patterns applied to /one/two/three/}
\label{table_greedy_example}
\begin{tabular}{l | p{12cm}}
Pattern & Matches \\ \hline \hline
\verb=/(.*)/= & one/two/three \\ \hline
\verb=/(.*?)/= & one \\ \hline
\end{tabular}
\end{table}

The first regex is greedy, and matches as much as it possibly can, going
out to the last slash. The second is non-greedy, and so stops as early as it can, when it encounter
the second slash.

\subsection{Making a match optional}
\index{Optional matching}
\index{?}

The ? character makes a single character match optional. This is extremely useful for 
common misspellings, or elements that may, or may not, appear in a string. For example, you 
might use it in a word when you're not sure whether it's supposed to be hyphenated:

\verb=e-?mail=

The above pattern matches both "email" and "e-mail", so that either
spelling will be accepted. Likewise, you could use:

\verb=colou?r=

to match the word color both as it is spelled in the USA, and the way
that it is spelled in the rest of the world.

Additionally, the \verb=?= character turns off the "greedy" nature of the \verb=+= 
and \verb=*= characters. Thus, putting a \verb=?= after a \verb=+= or a 
\verb=*= will make it match as little as it possibly can. See the earlier 
comments about greedy matching - \ref{greedy}.

Further examples of the greedy vs. non-greed behavior will follow once we have learned 
about backreferences.

\subsection{Grouping and capturing}
\index{( )}

Parentheses allow you to group several characters as a unit, and also to capture the results of 
a match for later use. The ability to treat several characters as a unit is extremely useful in 
pattern matching. The following example is functional, but not very useful:

\verb=(abc)+=

This will look for the sequence "abc" appearing one or more times, and so would match 
the string "abc" and the string "abcabc".

\label{backreferences}
\index{backreferences}
Even more useful is the "capturing" functionality of the parentheses. Once a pattern has 
matched, you often want to know what matched, so that you can use it later. This is usually 
referred to as "backreferences."

For example, you may be looking for a .gif file, as in the example above, and you really 
want to know what .gif file you matched. By capturing the filename with parentheses, you can 
use it later on:

\verb=(.*\.gif)$=

In the event that this pattern matches, we will capture the matching value in a special 
variable, \verb=$1=. (In some contexts, the variable may be called \verb=%1= instead.) If you have more 
than one set of parentheses, the second one will be captured to the variable \verb=$2=, the third to \verb=$3=, 
and so on. Only values up through \verb=$9= are available, however.  The reason for this is that \verb=$10= 
would be ambiguous. It might mean \verb=$1=, followed by a literal zero (0), or it might mean \verb=$10=.  
Rather than providing additional syntax to disambiguate this term, the designer of 
mod\_rewrite instead chose to only provide backreferences through \verb=$9=.

The exact way in which you can exploit this feature will be more obvious later, once we 
start looking at the RewriteRule directive in Chapter \ref{chapter_rewriterule}.

Consider these two patterns, applied to the string ``canadian''.

\begin{verbatim}
c(.*)n
c(.*?)n
\end{verbatim}

The first pattern will return with a value of "anadia" in \verb=$1=, while the second will return 
with \verb=$1= set to "a". When it is in greedy mode, the \verb=.*= will gobble up as much as it can, only 
stopping when it reaches the last n, but when in non-greedy mode, it will be satisfied with as 
little as possible, stopping with the first n it encounters.
It is instructive to acquire a tool such as Regex Coach, or Rebug, mentioned at the end of 
the chapter, and feed them these patterns and strings, to watch them match the different parts 
of the string. \underline{Mastering Regular Expressions} also has a very complete treatment of 
backreferences, greedy matching, and what actually happens during the matching phase.

\subsection{Character Classes}
\index{Character classes}
\index{[ ]}

A character class allows you to define a set of characters, and match any one of them. There 
are several built-in character classes, like the \verb=\s= metacharacter that you saw above. 
But using the \verb=[ ]= notation lets you define your own
custom character classes. As a very simple example, consider the following:

\verb=[abc]=

This character class matches the letter a, or the letter b, or the letter c. For example, if 
we wanted to match the subset of users whose usernames started with one of those letters, we 
might look for the pattern:

\verb=/home/([abc].*)=

This combines several of the characters that we've worked with. It ends up matching a 
directory path for that subset of users, and the username ends up in the \verb=$1= variable. Well, 
actually, not quite, as we'll see in a minute, but almost.

The character class syntax also allows you to specify a range of characters fairly easily. 
For example, if you wanted to match a number between 1 and 5, you can use the character 
class \verb=[1-5]=.

Within a character class, the \verb=^= character has special meaning, if it is the first character in 
the class. The character class \verb=[^abc]= is the opposite of the character class \verb=[abc]=. That is, it 
matches any character which is not a, b, or c.

Which brings us back to the example above, where we are attempting to match a 
username starting with a, b, or c. The problem with the example is that the * character is 
greedy, meaning that it attempts to match as much as it possibly can. If we want to force it to 
stop matching when it reaches a slash, we need to match only "not slash" characters:

\verb=/home/([abc][^/]+)=

I've replaced the \verb=.*= with \verb=[^/]+= which has the effect that, rather than matching any 
character, it matches only not-slash characters. In other words, it will only match up to a 
slash, or the end of the string, whichever comes first. Also, I've used \verb=+= instead of \verb=*=, since 
one-character usernames are typically not permitted. Now, \verb=$1= will contain the username, 
whereas, before, it could possibly have contained other directory path components after the 
username.

\subsection{Negation}
\index{!}

Finally, if you wish to negate an entire regular expression match, prefix it with !. This is not 
going to be consistent across all regular expression implementations, but can be used in a 
number of them. A very common use of this in the context of rewrite rules will be to indicate 
that you want a pattern to apply to all directories except for one. So, for example, if we 
wanted to exclude the /images directory from consideration, we would match the /images 
directory, but then negate the match, thus:

\verb=!^/images=

This matches any path not starting with /images. We'll see more of this kind of pattern 
match especially in Chapter \ref{chapter_proxy} on Proxying.

\section{Regex examples}

A few examples may be instructive in your understanding of how regular expressions 
work. We'll start with a few of the cases that you may frequently encounter, and suggest a 
few alternate solutions to each.

\subsection{Email address}
\index{Email address}

We'll start with a common favorite. You want to craft a regular expression that matches 
an email address. The general format of an email address is "something @ something . 
something". When you are crafting a regular expression from scratch, it's good to express the 
pattern to yourself in terms like this, because it's a good start towards writing the expression 
itself.

To express this as a regular expression, let's take the component parts. The catch all 
"something" part can likely be expressed as .*. The . and @ parts are literal characters.
So, this gives us something like:

\verb=.*@.*\..*=

This is a good start, and matches most email addresses. It will probably match all email 
addresses. However, it will also match a lot of stuff that isn't email addresses, like 
"@@@.@" and "@.com". So we have to try something a little more specific.

We want to require that the "something" before the @ sign is not zero length, and 
contains certain types of characters. For example, it should be alpha-numeric, but may also 
contain certain other special characters, like dot, underscore, or dash.

Fortunately, PCRE provides us with a convenient way to say "alpha-numeric 
characters,", using a named character class. There are quite a number of these, such as 
\verb=[:alpha:]= to match letters, \verb=[:digit:]= to match numbers 0 through 9, and \verb=[:alnum:]= to match 
alpha-numeric characters.

Next, we want to ensure that the domain name part of the pattern is alphanumeric too, 
except that the top level domain (tld), i.e., the last part of the domain name, must be letters. In 
the old days, we could have said it had to be three letters, but now there are a large number of 
perfectly valid domain names that don't match that requirement.

And we want to allow an arbitrary number of dots in the hostname, so that "a.com" and 
"mail.s.ms.uky.edu" are both valid hostname portions of an email address.
So we can say the above description as:

\begin{verbatim}
^[:alnum:]._-]@([:alnum:]+\.)+[:alpha:]+$
\end{verbatim}

This is far more specific, and will probably ensure a valid email address. There are still 
probably ways for it to match something that is not an email address, but it is unlikely.

However, you should note that there does not exist a regular expression
that matches all possible email addresses. Indeed, I started with
this example to give you a flavor for just how complicated it can be to
craft a pattern for something that is not well defined.

For more discussion of writing regular expressions to match email
addresses, simply search for ``email regex'' in your favorite search
engine, and you'll find many, many articles about how and why it is
impossible. 

\subsection{Phone number}
\index{Phone number}

Next we'll consider the problem of matching a phone number. This is much harder than it 
would at first appear. We'll assume, for the sake of simplicity, that we're just trying to match 
US phone numbers, which are 10 numbers.

The number consists of three numbers, then three more, then four more. These numbers 
may or may not be separated by a variety of things. The first three may or may not be 
enclosed in parentheses. So we'll try something like this:

\verb=\(?\d{3}\)?[-. ]?\d{3}[-. ]?\d{4}=

This pattern matches most US phone numbers, in most of the ordinary formats. The 
first three numbers may or may not be in parentheses, and the blocks of numbers may or may 
not be separated by dashes (-), dots (.) or spaces.

It is still far from foolproof, because users will come up with ways to submit data in 
unexpected format.

Let's go though the rule one metacharacter at a time:

\verb=\(?= - This metacharacter represents an optional opening parenthesis. The backslash is 
necessary because parentheses have special meaning, as discussed above. We want to remove 
that special meaning, and have a literal opening parenthesis. The question mark makes this 
character options. That is, the person entering the data may or may not enclose the first three 
numbers with parenthesis, and we want to ensure that either one is acceptable.

\verb=\d{3}= - This metacharacter introduces two objects that have not been seen so far. \d means 
a digit. d for digit. This can also be written as \verb=[:digit:]=, but the \verb=\d= notation tends to be more 
common, for the simple reason that it's less to type. The \verb={3}= following the \verb=\d= indicates that 
we want to math the character exactly three times. That is, we require three digits in this 
portion of the match, or it will return failure.

The \verb={n}= notation has two other possible syntaxes, if the number of characters is not 
known for certain ahead of time. These syntaxes are shown in the table below:

\begin{table}[ht]
\caption{Repetition syntax}
\begin{tabular}{l | p{12cm}}
Syntax &
Meaning\\ \hline \hline
\verb={n}= &
Requires that the character appear exactly n times \\ \hline
\verb={n,}= &
Requires that the character appear at least n times, but more are permitted. \\ \hline
\verb={n,m}= &
The character must appear at least n times, but not more than m times\\
\hline
\end{tabular}
\end{table}

\verb=\)?= - Like the opening parenthesis we started with, this is an optional closing parenthesis.

\verb=[-. ]?= - Another optional character, this allows, but does not require, a dash, a dot, or a 
space, to appear between the first three numbers and the next three numbers.
The rest of the expression is exactly the same as what we have already done, except that 
the last block of numbers contains 4 numbers, rather than three.

The next step in crafting a regular expression is to think of the ways in which your 
pattern will break, and whether it is worth the additional work to catch these edge cases. For 
example, some users will enter a 1 before the entire number. Some phone numbers will have 
an extension number on the end. And that one hard-to-please user will insist on separating the 
numbers with a slash rather than one of the characters we have specified. These can probably 
be solved with a more complex regex, but the increased complexity comes at the price of 
speed, as well as a loss of readability. It took a page to explain what the current regex does, 
and that's at least some indication of how much time it would take you to decipher a regex 
when you come back to it in a few months and have forgotten what it is supposed to be 
doing.

\subsection{Matching URIs}

Finally, since this is, after all, a book about mod\_rewrite, it seems reasonable to give 
some examples of matching URIs, as that is what you will primarily be doing for the rest of 
the book.

Most of the directives that we will discuss in the remainder of the book, take regular 
expressions as one of their arguments. And, much of the time, those regular expressions will 
describe a URI, which is the technical term for the resource that was requested from your 
server. And most of the time, that means everything after the http://www.domain.com part of the 
web address.

I'll give several common examples of things that you might want to match.

\subsubsection{Matching the homepage}

Very frequently, people will want to match the home page of the website. Typically, that 
means that the requested URI is either nothing at all, or is /, or is some index page such as 
/index.html or /index.php. The case where it is nothing at all would be when the requested 
address was http://www.example.com with no trailing slash.

First, I'll consider the case where they request either http://www.example.com or 
http://www.example.com/ (ie, with or without the trailing slash, but with no file requested). In 
other words, we want to match an optional slash. 

As you probably remember from above, you use the ? character to make a match 
optional. Thus, we have: \verb=^/?$ =

This matches a string that starts with, and ends with, an optional slash. Or, stated 
differently, it matches either something that starts ends with a slash, or something that starts 
and ends with nothing.

Next, we introduce the additional complexity of the file name. That is, we want to match 
any of the following four strings:

\begin{itemize}
\item The empty string - that is, they requested http://www.example.com with no trailing slash.
\item / - they requested http://www.example.com/ with a trailing slash.
\item /index.html
\item /index.php
\end{itemize}

We'll build on the regex that we had last time, and get \verb=^/?(index.(html|php))?$=
This isn't quite right, as you'll see in a moment, but it's mostly right. It does, however, 
introduce a new syntax that hasn't been mentioned heretofore. That is the | syntax, which has 
the fancy name of "alternation" and means "one or the other." So (html|php) means "either 
'html' or 'php'."

So, we've got a regex that means a string that starts with a slash (optional) followed by 
index., followed by either html or php, and that entire string (starting with the index) is also 
optional, and then the string ends.

The one problem with this regex is that it also matches the strings 'index.php' and 
'index.html', without a leading slash. While, strictly speaking, this is incorrect, in the actual 
context of matching a URI, it is probably not of any great concern. Although a client could in 
fact request one of these two values, for one thing, they are rather unlikely to do so, but, for 
another, even if they do, it's probably ok to treat them as though they had requested a valid 
URI.

\subsubsection{Matching a directory}
\index{Directory}

If you wanted to find out what directory a particular requested URI was in, or, perhaps, 
what keyword it started with, you need to match everything up to the first slash. This will 
look something like the following: 

\verb=^/([^/]+)=

This regex has a number of components. First, there's the standard \verb=^/= which we'll see a 
lot, meaning "starts with a slash." Following that, we have the character class \verb=[^/]=, which will 
match any "not slash" character. This is followed by a + indicating that we want one or more 
of them, and enclosed in parentheses so that we can have the value for later observation, in 
\$1.

\subsubsection{Matching a filetype}
\index{File type}

For the third example, we'll try to match everything that has a particular file extension. 
This, too, is a very common need. For example, we want to match everything that is an image 
file. The following regex will do that, for the most common image types:

\verb#\.(jpg|gif|png)$#

Later on, you'll see how to make this case insensitive, so that files with upper-case file 
extensions are also matched.

\chapter{Regex tools}
\index{Regex testers}

If you're going to spend more than just a little time messing with regexes, you're 
eventually going to want a tool that helps you visualize what's going on. There are a number 
of them available, each of which has different strengths and weaknesses. You'll find that 
most of the really good tools for regular expression development come out of the Perl 
community, where regular expressions are particularly popular, and tend to get used in 
almost every program.

\section{Regex Coach}
\index{Regex Coach}
\label{regexcoach}

Regex Coach is available for Windows and Linux, 
and can be downloaded from \verb=http://www.weitz.de/regex-coach=. 
Regex Coach allows you to step through a regular expression and watch
what it does and does not match. This can be extremely instructive in
learning to write your own regular expressions.

TODO SCREENSHOT

Regex Coach is free, but it is not Open Source.

\section{Reggy}
\index{Reggy}
\label{reggy}

Reggy is a Mac OS X application that provides a simple interface for
crafting and testing regular expressions. It will identify what parts of
a string are matched by your regular expression.

Reggy is available at \verb=http://code.google.com/p/reggy/= and is
licensed under the GPL.

TODO SCREENSHOT

\section{pcretest}
\index{pcretest}
\label{pcretest}

pcretest is a command-line regular expression tester that is available
on most distributions of Linux, where it is usually installed by
default.

In addition to simply telling you whether a particular string matched or
not, it will also tell you what each of the various backreferences will
be set to.

In TODO SCREENSHOT you can see what each of the various backreferences
will be set to once the regular expression has been evaluated.

\section{Visual Regexp}
\index{Visual Regexp}
\label{visualregexp}

Visual Regexp, available at \verb=http://laurent.riesterer.free.fr/regexp/=, has more features
than the options listed above, and might be a good option once you have
mastered the basics of regular expressions and are ready to move onto
something a little more sophisticated. It shows backreferences, and
offers a wide variety of suggestions to help build a regex.

Visual Regexp is available as a Windows executable or as a Tcl/Tk
script. TODO SCREENSHOT

\section{Regular Expression Tester}
\index{Regular Expression Tester}

Rather than being a stand-alone application like the others listed
above, this is a Firefox plugin. It's available at
\verb#https://addons.mozilla.org/en-US/firefox/addon/2077#, and, of
course, requires Firefox to work.

\subsection{Online tools}
\index{Online regex testers}

In addition to these tools, there are many online tools, which you can
use without having to download or install anything. These are of a wide
variety of feature sets and quality, so I'd encourage you to shop around
a little to find one that seems to work well. These appear and disappear
on a weekly basis, and so I can't promise that these sites will still
be available at the time that you read this, but here are some that are
worth mentioning at the time of writing:

\subsubsection{RegExr}
\index{RegExr}

\verb#http://gskinner.com/RegExr/# - Includes a variety of pre-defined
character classes, and the ability to save your regular expressions for
later reference. Requires Javascript to use.

\subsubsection{Regex Pal}
\index{Regex Pal}



\verb#http://regexpal.com/# - Less full-featured than RegExr, but
sufficient for the purpose of crafting and testing regular expressions
for the purpose of mod\_rewrite, which doesn't require replace
functionality or multi-line matches.

\section{RewriteRule generators}

You may find various websites that purport to be RewriteRule generators.
I strongly encourage you to avoid these, and instead to learn how to
craft your own rules. I've evaluated several of these sites, and every
one has resulted in RewriteRule directives that were either enormously
inefficient, or completely wrong.

\section{Summary}

Having a good grasp of Regular Expressions is a necessary prerequisite to working with 
mod\_rewrite. All too often, people try to build regular expressions by the brute-force method, 
trying various different combinations at random until something seems to mostly work. This 
results in expressions that are inefficient and fragile, as well as a great waste of time, and 
much frustration.

Keep a bookmark in this chapter, and refer back to it when you're trying to figure out 
what a particular regex is doing.

Other recommended reference sources include the Perl regular expression documentation, 
which you can find online at \verb=http://www.perldoc.com/perl5.8.0/pod/perlre.html= or by typing 
\verb=perldoc perlre= at your command line, and the PCRE documentation, which you can find online at 
\verb=http://pcre.org/pcre.txt=. 

