\part{mod\_rewrite}
\label{partrewrite}

\chapter{Introduction to mod\_rewrite}
\label{rewrite_intro}

mod\_rewrite is the power tool of Apache httpd URL mapping. Of course, sometimes you just need a screwdriver, but when you need the power tool, it's good to know where to find it.

mod\_rewrite provides sophisticated URL via regular expressions, and the ability to do a variety of transformations,including, but not limited to, modification of the request URL. You can additionally return a variety of status codes, set cookies and environment variables, proxy requests to another server, or send redirects to the client.

In this chapter we'll cover mod\_rewrite syntax and usage, and in the next chapter we'll give a variety of examples of using mod\_rewrite in common scenarios.

\section{Loading mod\_rewrite}

To use mod\_rewrite in any context, you need to have the module loaded. If you're the server administrator, this means having the following line somewhere in your Apache httpd configuration:

\verb~LoadModule rewrite_module modules/mod_rewrite.so~

This tells httpd that it needs to load mod\_rewrite at startup time, so as to make its functionality available to your configuration files.

If you are not the server administrator, then you'll need to ask your server administrator if the module is available, or experiment to see if it is. If you're not sure, you can test to see whether it's enabled in the following manner.

Create a subdirectory in your document directory. Let's call it \verb~test_rewrite~

Create a file in that directory called \verb~.htaccess~ and put the following text in it:

\verb~RewriteEngine on~

Create another file in that directory called \verb~index.html~ containing the following text:

\begin{verbatim}
<html>
Hello, mod_rewrite
</html>
\end{verbatim}

Now, point your browser at that location:

\verb~http://example.com/test_rewrite/index.html~

You'll see one of two things. Either you'll see the words \verb~Hello, mod_rewrite~ in your browser, or you'll see the ominous words \verb~Internal Server Error~. In the former case, everything is fine - mod\_rewrite is loaded and your \verb~.htacces~ file worked just fine. If you got an \verb~Internal Server Error~, that was httpd complaining that it didn't know what to do with the \verb~RewriteEngine~ directive, because mod\_rewrite wasn't loaded.

If you have access to the server's error log file, you'll see the following in it:

\begin{verbatim}
Invalid command 'RewriteEngine', perhaps misspelled or defined by a module not included in the server configuration
\end{verbatim}

Which is httpd's way of saying that you used a directive (\verb~RewriteEngine~) without first loading the module that defines that directive.

If you see the \verb~Internal Server Error~ message, or that log file message, it's time to contact your server administrator and ask if they'll load mod\_rewrite for you.

However, this is fairly unlikely, since mod\_rewrite is a fairly standard part of any Apache http server's bag of tricks.

\section{RewriteEngine}

In the section above, we used the \verb~RewriteEngine~ directive without defining what it does.

The \verb~RewriteEngine~ directive enables or disables the runtime rewriting engine. The directive defaults to \verb~off~, so the result is that rewrite directives will be ignored in any scope where you don't have the following:

\verb~RewriteEngine On~

While we won't always include that in every example in this book, it should be assumed, from this point forward, that every use of mod\_rewrite occurs in a scope where \verb~RewriteEngine~ has been turned on.

\section{mod\_rewrite in .htaccess files}
\index{htaccess files}
\index{mod\_rewrite in .htacess files}

Before we go any further, it's critical to note that things are different, in several important ways, if you have to use .htaccess files for configuration.

\subsection{What are .htaccess files?}

.htaccess files are per-directory configuration files, for use by people who don't have access to the main server configuration file. For the most part, you put configuration directives into .htaccess files just as you would in a \verb~<Directory>~ block in the server configuration, but there are some differences.

The most important of these differences is that the .htaccess file is consulted every time a resource is requested from the directory in question, whereas configurations placed in the main server configuration file are loaded once, at server startup. 

The positive side of this is that you can modify the contents of a .htaccess file and have the change take effect immediately, as of the next request received by the server.

The negative is that the .htaccess file needs to be loaded from the filesystem on every request, resulting in an incremental slowdown for every request. Additionally, because httpd doesn't know ahead of time what directories contain .htaccess files, it has to look in each directory for them, all along the path to the requested resource, which results in a slowdown that grows with the depth of the directory tree.

In Apache httpd 2.2 and earlier, .htaccess files are enabled by default - that is the configuration directive that enables them, \verb~AllowOverride~, has a default value of \verb~All~. In 2.4 and later, it has a default value of \verb~None~, so .htaccess files are disabled by default.

A typical configuration to permit the use of .htaccess files looks like:

\begin{verbatim}
<Directory />
	AllowOverride None
</Directory>

DocumentRoot /var/www/html
<Directory /var/www/html>
	AllowOverride All
</Directory /var/www/html>
\end{verbatim}

That is to say, .htaccess files are disallowed for the entire filesystem, 
starting at the root, but then are permitted in the document directories.
This prevents httpd from looking for .htaccess files in \verb~/~, \verb~/var~, 
and \verb~/var/www~ on the way to looking in \verb~/var/www/html~.

\subsection{Ok, so, what's the deal with mod\_rewrite in .htaccess files?}

There are two major differences that you must be aware of before we proceed any further. The exact implications of these differences will become more apparent as we go, but I wouldn't want them to surprise you.

First, there are two directives that you cannot use in .htaccess files. These directives are \verb~RewriteMap~ and \verb~RewriteLog~. These must be defined in the main server configuration. The reasons for this will be discussed in greater length when we get to the sections about those directives (\ref{rewritemap} and \ref{rewritelogging}, respectively.).

Second, and more importantly, the syntax of \verb~RewriteRule~ directives changes in .htaccess context in a way that you'll need to be aware of every time you write a \verb~RewriteRule~. Specifically, the directory path that you're in will be removed from the URL path before it is presented to the \verb~RewriteRule~.

The exact implications of this will become clearer as we show you examples. And, indeed, every example in this book will be presented in a form for the main config, and a form for .htaccess files, whenever there is a difference between the two forms. But we'll start with a simple example to illustrate the idea.

Some of this, you'll need to take on faith at the moment, since we've not yet introduced several of the concepts presented in this example, so please be patient for now.

Consider a situation where you want to apply a rewrite to content in the \verb~/images/puppies/~ subdirectory of your website. You have four options: You can put the \verb~RewriteRule~ in the main server configuration file; You can place it in a .htacess file in the root of your website; You can place it in a .htaccess file in the \verb~images~ directory; Or you can place it in a .htaccess file in the \verb~images/puppies~ directory.

Here's what the rule might look like in those various scenarios:

\begin{tabular}{|l|l|}
\hline \bf{Location} & \bf{Rule} \\ 
\hline Main config & \verb~RewriteRule ^/images/puppies/(.*).jpg /dogs/$1.gif~ \\
\hline Root directory &  \verb~RewriteRule ^images/puppies/(.*).jpg /dogs/$1.gif~ \\ 
\hline images directory &  \verb~RewriteRule ^puppies/(.*).jpg /dogs/$1.gif~ \\ 
\hline images/puppies directory &  \verb~RewriteRule ^(.*).jpg /dogs/$1.gif~ \\  
\hline 
\end{tabular} 

For the moment, don't worry too much about what the individual rules do.
Look instead at the URL path that is being considered in each rule, and
notice that for each directory that a .htaccess file is placed in, the directory path that \verb~RewriteRule~ may consider is relative to that directory, and anything above that becomes invisible for the purpose of mod\_rewrite.

Don't worry too much if this isn't crystal clear at this point. It will become more clear as we proceed and you see more examples.

\subsection{So, what do I do?}

If you don't have access to the main server configuration file, as it the case for many of the readers of this book, don't despair. mod\_rewrite is still a very powerful tool, and can be persuaded to do almost anything that you need it to do. You just need to be aware of its limitations, and adjust accordingly when presented with an example rule.

We aim to help you do that at each step along this journey.

\section{RewriteOptions}

\chapter{RewriteRule}
\label{chapter_rewriterule}

We're going to start with the \verb~RewriteRule~ directive, as it is the workhorse of mod\_rewrite, and the directive that you'll encounter most frequently.

\verb~RewriteRule~ performs manipulation of a requested URL, and along the way can do a number of additional things.

The syntax of a \verb~RewriteRule~ is fairly simple, but you'll find that exploring all of the possible permutations of it will take a while. So we'll provide a lot of examples along the way to illustrate.

If you learn best by example, you may want to jump back and forth between this section and \ref{rewriteexamples} to help you make sense of this all.

\section{Syntax}

A \verb~RewriteRule~ directive has two required directives and optional flags. It looks like:

\begin{verbatim}
RewriteRule PATTERN TARGET [FLAGS]
\end{verbatim}

The following sections will discuss each of those arguments in great detail.

\section{Pattern}

The \verb~PATTERN~ argument of the \verb~RewriteRule~ is a regular expression that is applied to the URL path, or file path, depending on the context.

In VirtualHost context, or in server-wide context, \verb~PATTERN~ will be matched against the part of the URL after the hostname and port, and before the query string. For example, in the URL \verb~http://example.com/dogs/index.html?dog=collie~, the pattern will be matched against \verb~/dogs/index.html~.

In Directory and htaccess context, \verb~PATTERN~ will be matched against the filesystem path, after removing the prefix that led the server to the current \verb~RewriteRule~ (e.g. either "dogs/index.html" or "index.html" depending on where the directives are defined).

Subsequent \verb~RewriteRule~ patterns are matched against the output of the last matching \verb~RewriteRule~.

It is assumed, at this point, that you've already read Chapter \ref{chapter_regex}, and/or are familiar with what a regular expression is, and how to craft one.

\section{Target}

The target of a \verb~RewriteRule~ can be one of the following:

\subsection{A file-system path}

Designates the location on the file-system of the resource to be delivered to the client. Substitutions are only treated as a file-system path when the rule is configured in server (virtualhost) context and the first component of the path in the substitution exists in the file-system

\subsection{URL-path}

A DocumentRoot-relative path to the resource to be served. Note that mod\_rewrite tries to guess whether you have specified a file-system path or a URL-path by checking to see if the first segment of the path exists at the root of the file-system. For example, if you specify a Substitution string of \verb~/www/file.html~, then this will be treated as a URL-path unless a directory named www exists at the root or your file-system (or, in the case of using rewrites in a .htaccess file, relative to your document root), in which case it will be treated as a file-system path. If you wish other URL-mapping directives (such as Alias) to be applied to the resulting URL-path, use the \verb~[PT]~ flag as described below.

\subsection{Absolute URL}

If an absolute URL is specified, mod\_rewrite checks to see whether the hostname matches the current host. If it does, the scheme and hostname are stripped out and the resulting path is treated as a URL-path. Otherwise, an external redirect is performed for the given URL. To force an external redirect back to the current host, see the \verb~[R]~ flag below.

\subsection{- (dash)}

A dash indicates that no substitution should be performed (the existing path is passed through untouched). This is used when a flag (see below) needs to be applied without changing the path.

\section{Flags}
\label{flags}
\index{Rewrite flags}
\index{flags}

Flags modify the behavior of the rule. You may have zero or more flags, and the effect is cumulative. Flags may be repeated where appropriate. For example, you may set several environment variables by using several \verb~[E]~ flags, or set several cookies with multiple \verb~[CO]~ flags. Flags are separated with commas:

\begin{verbatim}
[B,C,NC,PT,L]
\end{verbatim}

There are a \emph{lot} of flags. Here they are:

\subsection{B - escape backreferences}
\index{B flag}
\index{Rewrite flags!B}
\label{bflag}

The \verb~[B]~ flag instructs \verb~RewriteRule~ to escape non-alphanumeric characters before applying the transformation.

mod\_rewrite has to unescape URLs before mapping them, so backreferences are unescaped at the time they are applied. Using the B flag, non-alphanumeric characters in backreferences will be escaped. (See \ref{backreferences} for discussion of backreferences.) For example, consider the rule:

\begin{verbatim}
RewriteRule ^search/(.*)$ /search.php?term=$1
\end{verbatim}

Given a search term of \verb~'x & y/z'~, a browser will encode it as \verb~'x%20%26%20y%2Fz'~, making the request \verb~'search/x%20%26%20y%2Fz'~. Without the B flag, this rewrite rule will map to \verb~'search.php?term=x & y/z'~, which isn't a valid URL, and so would be encoded as \verb~search.php?term=x%20&y%2Fz=~, which is not what was intended.

With the B flag set on this same rule, the parameters are re-encoded before being passed on to the output URL, resulting in a correct mapping to \verb~/search.php?term=x%20%26%20y%2Fz~.

Note that you may also need to set \verb~AllowEncodedSlashes~ to On to get this particular example to work, as httpd does not allow encoded slashes in URLs, and returns a 404 if it sees one.

This escaping is particularly necessary in a proxy situation, when the backend may break if presented with an unescaped URL.

\subsection{C - chain}
\index{C flag}
\index{Rewrite flags!C}
\label{cflag}

The \verb~[C]~ or \verb~[chain]~ flag indicates that the RewriteRule is chained to the next rule. That is, if the rule matches, then it is processed as usual and control moves on to the next rule. However, if it does not match, then the next rule, and any other rules that are chained together, will be skipped.

\subsection{CO - cookie}
\index{CO flag}
\index{Rewrite flags!CO}
\label{coflag}

The \verb~[CO]~, or \verb~[cookie]~ flag, allows you to set a cookie when a particular RewriteRule matches. The argument consists of three required fields and four optional fields.

The full syntax for the flag, including all attributes, is as follows:

\begin{verbatim}
[CO=NAME:VALUE:DOMAIN:lifetime:path:secure:httponly]
\end{verbatim}

You must declare a name, a value, and a domain for the cookie to be set.

\subsubsection{Domain}

The domain for which you want the cookie to be valid. This may be a hostname, such as www.example.com, or it may be a domain, such as .example.com. It must be at least two parts separated by a dot. That is, it may not be merely .com or .net. Cookies of that kind are forbidden by the cookie security model.
You may optionally also set the following values:

\subsubsection{Lifetime}

The time for which the cookie will persist, in minutes.
A value of 0 indicates that the cookie will persist only for the current browser session. This is the default value if none is specified.

\subsubsection{Path}

The path, on the current website, for which the cookie is valid, such as \verb~/customers/~ or \verb~/files/download/~.
By default, this is set to \verb~/~ - that is, the entire website.

\subsubsection{Secure}

If set to secure, true, or 1, the cookie will only be permitted to be translated via secure (https) connections.

\subsubsection{httponly}

If set to HttpOnly, true, or 1, the cookie will have the HttpOnly flag set, which means that the cookie will be inaccessible to JavaScript code on browsers that support this feature.

\subsubsection{Example}

Consider this example:

\begin{verbatim}
RewriteEngine On
RewriteRule ^/index\.html - [CO=frontdoor:yes:.example.com:1440:/]
\end{verbatim}

In the example give, the rule doesn't rewrite the request. The '-' rewrite target tells mod\_rewrite to pass the request through unchanged. Instead, it sets a cookie called 'frontdoor' to a value of 'yes'. The cookie is valid for any host in the .example.com domain. It will be set to expire in 1440 minutes (24 hours) and will be returned for all URIs (i.e., for the path '/').

\subsection{DPI - discardpath}
\index{Rewrite flags!DPI}
\index{DPI flag}
\label{dpiflag}

The DPI flag causes the \verb~PATH_INFO~ portion of the rewritten URI to be discarded.

This flag is available in version 2.2.12 and later.

In per-directory context, the URI each \verb~RewriteRule~ compares against is the concatenation of the current values of the URI and \verb~PATH_INFO~.

The current URI can be the initial URI as requested by the client, the result of a previous round of mod\_rewrite processing, or the result of a prior rule in the current round of mod\_rewrite processing.

In contrast, the \verb~PATH_INFO~ that is appended to the URI before each rule reflects only the value of \verb~PATH_INFO~ before this round of mod\_rewrite processing. As a consequence, if large portions of the URI are matched and copied into a substitution in multiple \verb~RewriteRule~ directives, without regard for which parts of the URI came from the current \verb~PATH_INFO~, the final URI may have multiple copies of \verb~PATH_INFO~ appended to it.

Use this flag on any substitution where the \verb~PATH_INFO~ that resulted from the previous mapping of this request to the filesystem is not of interest. This flag permanently forgets the \verb~PATH_INFO~ established before this round of mod\_rewrite processing began. \verb~PATH_INFO~ will not be recalculated until the current round of mod\_rewrite processing completes. Subsequent rules during this round of processing will see only the direct result of substitutions, without any \verb~PATH_INFO~ appended.

\subsection{E - env}
\index{E flag}
\index{Rewrite flags!E}
\label{eflag}

With the \verb~[E]~, or \verb~[env]~ flag, you can set the value of an environment variable. Note that some environment variables may be set after the rule is run, thus unsetting what you have set.

The full syntax for this flag is:

\begin{verbatim}
[E=VAR:VAL] [E=!VAR]
\end{verbatim}

VAL may contain backreferences (See section \ref{backreferences}) (\verb~$N~ or \verb~%N~) which will be expanded.

Using the short form

\begin{verbatim}
[E=VAR]
\end{verbatim}

you can set the environment variable named VAR to an empty value.

The form

\begin{verbatim}
[E=!VAR]
\end{verbatim}

allows to unset a previously set environment variable named VAR.

Environment variables can then be used in a variety of contexts, including CGI programs, other RewriteRule directives, or CustomLog directives.

The following example sets an environment variable called 'image' to a value of '1' if the requested URI is an image file. Then, that environment variable is used to exclude those requests from the access log.

\begin{verbatim}
RewriteRule \.(png|gif|jpg)$ - [E=image:1]
CustomLog logs/access_log combined env=!image
\end{verbatim}

Note that this same effect can be obtained using SetEnvIf. This technique is offered as an example, not as a recommendation.

The \verb~[E]~ flag may be repeated if you want to set more than one environment variable at the same time:

\begin{verbatim}
RewriteRule \.pdf$ [E=document:1,E=pdf:1,E=done]
\end{verbatim}

\subsection{END}
\label{endflag}
\index{Rewrite flags!END}
\index{END flag}

Although the flags are presented here in alphabetical order, it makes more sense to go read the section about the L flag first (\ref{lflag}) and then come back here.

Using the \verb~[END]~ flag terminates not only the current round of rewrite processing (like \verb~[L]~) but also prevents any subsequent rewrite processing from occurring in per-directory (htaccess) context.

This does not apply to new requests resulting from external redirects.

\subsection{F - forbidden}
\label{fflag}
\index{Rewrte flags!F}
\index{F flag}

Using the \verb~[F]~ flag causes the server to return a 403 Forbidden status code to the client. While the same behavior can be accomplished using the Deny directive, this allows more flexibility in assigning a Forbidden status.

The following rule will forbid \verb~.exe~ files from being downloaded from your server.

\begin{verbatim}
RewriteRule \.exe - [F]
\end{verbatim}

This example uses the "-" syntax for the rewrite target, which means that the requested URI is not modified. There's no reason to rewrite to another URI, if you're going to forbid the request.

When using \verb~[F]~, an \verb~[L]~ is implied - that is, the response is returned immediately, and no further rules are evaluated.

\subsection{G - gone}
\label{gflag}
\index{G flag}
\index{Rewrite flags!G}

The \verb~[G]~ flag forces the server to return a 410 Gone status with the response. This indicates that a resource used to be available, but is no longer available.

As with the \verb~[F]~ flag, you will typically use the "-" syntax for the rewrite target when using the \verb~[G]~ flag:

\begin{verbatim}
RewriteRule oldproduct - [G,NC]
\end{verbatim}

When using \verb~[G]~, an \verb~[L]~ is implied - that is, the response is returned immediately, and no further rules are evaluated.

\subsection{H - handler}
\label{hflag}
\index{H flag}
\index{Rewrite flags!H}

Forces the resulting request to be handled with the specified handler. For example, one might use this to force all files without a file extension to be parsed by the php handler:

\begin{verbatim}
RewriteRule !\. - [H=application/x-httpd-php]
\end{verbatim}

The regular expression above - \verb~!\.~ - will match any request that does not contain the literal . character.

This can be also used to force the handler based on some conditions. For example, the following snippet used in per-server context allows .php files to be displayed by mod\_php if they are requested with the .phps extension:

\begin{verbatim}
RewriteRule ^(/source/.+\.php)s$ $1 [H=application/x-httpd-php-source]
\end{verbatim}

The regular expression above - \verb~^(/source/.+\.php)s$~ - will match any request that starts with \verb~/source/~ followed by 1 or n characters followed by \verb~.phps~ literally. The backreference \verb~$1~ referrers to the captured match within parenthesis of the regular expression.

\subsection{L - last}
\label{lflag}
\index{L flag}
\index{Rewrite flags!L}

The \verb~[L]~ flag causes mod\_rewrite to stop processing the rule set. In most contexts, this means that if the rule matches, no further rules will be processed. This corresponds to the last command in Perl, or the break command in C. Use this flag to indicate that the current rule should be applied immediately without considering further rules.

If you are using \verb~RewriteRule~ in either .htaccess files or in \verb~<Directory>~ sections, it is important to have some understanding of how the rules are processed. The simplified form of this is that once the rules have been processed, the rewritten request is handed back to the URL parsing engine to do what it may with it. It is possible that as the rewritten request is handled, the .htaccess file or \verb~<Directory>~ section may be encountered again, and thus the ruleset may be run again from the start. Most commonly this will happen if one of the rules causes a redirect - either internal or external - causing the request process to start over.

It is therefore important, if you are using \verb~RewriteRule~ directives in one of these contexts, that you take explicit steps to avoid rules looping, and not count solely on the \verb~[L]~ flag to terminate execution of a series of rules, as shown below.

An alternative flag, \verb~[END]~, can be used to terminate not only the current round of rewrite processing but prevent any subsequent rewrite processing from occurring in per-directory (htaccess) context. This does not apply to new requests resulting from external redirects.

The example given here will rewrite any request to index.php, giving the original request as a query string argument to \verb~index.php~, however, the \verb~RewriteCond~ ensures that if the request is already for index.php, the \verb~RewriteRule~ will be skipped.

\begin{verbatim}
RewriteBase /
RewriteCond %{REQUEST_URI} !=/index.php
RewriteRule ^(.*) /index.php?req=$1 [L,PT]
\end{verbatim}

\subsection{N - next}
\label{nflag}
\index{N flag}
\index{Rewrite flags!N}

The \verb~[N]~ flag causes the ruleset to start over again from the top, using the result of the ruleset so far as a starting point. Use with extreme caution, as it may result in loop.

The \verb~[N]~ flag could be used, for example, if you wished to replace a certain string or letter repeatedly in a request. The example shown here will replace A with B everywhere in a request, and will continue doing so until there are no more As to be replaced.

\begin{verbatim}
RewriteRule (.*)A(.*) $1B$2 [N]
\end{verbatim}

You can think of this as a while loop: While this pattern still matches (i.e., while the URI still contains an A), perform this substitution (i.e., replace the A with a B).

\subsection{NC - nocase}
\label{ncflag}
\index{NC flag}
\index{Rewrite flags!NC}

Use of the \verb~[NC]~ flag causes the \verb~RewriteRule~ to be matched in a case-insensitive manner. That is, it doesn't care whether letters appear as upper-case or lower-case in the matched URI.

In the example below, any request for an image file will be proxied to your dedicated image server. The match is case-insensitive, so that .jpg and .JPG files are both acceptable, for example.

\begin{verbatim}
RewriteRule (.*\.(jpg|gif|png))$ http://images.example.com$1 [P,NC]
\end{verbatim}

\subsection{NE - noescape}
\label{neflag}
\index{NE flag}
\index{Rewrite flag!NE}

By default, special characters, such as \verb~\&~ and \verb~?~, for example, will be converted to their hexcode equivalent. Using the \verb~[NE]~ flag prevents that from happening.

\begin{verbatim}
RewriteRule ^/anchor/(.+) /bigpage.html#$1 [NE,R]
\end{verbatim}

The above example will redirect \verb~/anchor/xyz~ to \verb~/bigpage.html#xyz~. Omitting the \verb~[NE]~ will result in the \verb~#~ being converted to its hexcode equivalent, \verb~%23~, which will then result in a 404 Not Found error condition.

\subsection{NS - nosubreq}
\label{nsflag}
\index{NS flag}
\index{Rewrite flag!NS}

Use of the \verb~[NS]~ flag prevents the rule from being used on subrequests. For example, a page which is included using an SSI (Server Side Include) is a subrequest, and you may want to avoid rewrites happening on those subrequests. Also, when mod\_dir tries to find out information about possible directory default files (such as index.html files), this is an internal subrequest, and you often want to avoid rewrites on such subrequests. On subrequests, it is not always useful, and can even cause errors, if the complete set of rules are applied. Use this flag to exclude problematic rules.

To decide whether or not to use this rule: if you prefix URLs with CGI-scripts, to force them to be processed by the CGI-script, it's likely that you will run into problems (or significant overhead) on sub-requests. In these cases, use this flag.

Images, javascript files, or css files, loaded as part of an HTML page, are not subrequests - the browser requests them as separate HTTP requests.

\subsection{P - proxy}
\label{pflag}
\index{P flag}
\index{Rewrite flag!P}

Use of the \verb~[P]~ flag causes the request to be handled by mod\_proxy, and handled via a proxy request. For example, if you wanted all image requests to be handled by a back-end image server, you might do something like the following:

\begin{verbatim}
RewriteRule /(.*)\.(jpg|gif|png)$ http://images.example.com/$1.$2 [P]
\end{verbatim}

Use of the \verb~[P]~ flag implies \verb~[L]~. That is, the request is immediately pushed through the proxy, and any following rules will not be considered.

You must make sure that the substitution string is a valid URI (typically starting with \verb~http://hostname~ ) which can be handled by the mod\_proxy. If not, you will get an error from the proxy module. Use this flag to achieve a more powerful implementation of the \verb~ProxyPass~ directive, to map remote content into the namespace of the local server.

\subsubsection{Security Warning}

Take care when constructing the target URL of the rule, considering the security impact from allowing the client influence over the set of URLs to which your server will act as a proxy. Ensure that the scheme and hostname part of the URL is either fixed, or does not allow the client undue influence.

\subsubsection{Performance warning}

Using this flag triggers the use of mod\_proxy, without handling of persistent connections. This means the performance of your proxy will be better if you set it up with \verb~ProxyPass~ or \verb~ProxyPassMatch~

This is because this flag triggers the use of the default worker, which does not handle connection pooling.
Avoid using this flag and prefer those directives, whenever you can.

Note: mod\_proxy must be enabled in order to use this flag.

See Chapter \ref{chapter_proxy} for a more thorough treatment of proxying.

\subsection{PT - passthrough}
\label{ptflag}
\index{PT flag}
\index{Rewrite flag!PT}

The target (or substitution string) in a \verb~RewriteRule~ is assumed to be a file path, by default. The use of the \verb~[PT]~ flag causes it to be treated as a URI instead. That is to say, the use of the \verb~[PT]~ flag causes the result of the \verb~RewriteRule~ to be passed back through URL mapping, so that location-based mappings, such as \verb~Alias~, \verb~Redirect~, or \verb~ScriptAlias~, for example, might have a chance to take effect.

If, for example, you have an \verb~Alias~ for \verb~/icons~, and have a \verb~RewriteRule~ pointing there, you should use the \verb~[PT]~ flag to ensure that the \verb~Alias~ is evaluated.

\begin{verbatim}
Alias /icons /usr/local/apache/icons
RewriteRule /pics/(.+)\.jpg$ /icons/$1.gif [PT]
\end{verbatim}

Omission of the \verb~[PT]~ flag in this case will cause the \verb~Alias~ to be ignored, resulting in a 'File not found' error being returned.

The \verb~[PT]~ flag implies the \verb~[L]~ flag: rewriting will be stopped in order to pass the request to the next phase of processing.

Note that the \verb~[PT]~ flag is implied in per-directory contexts such as \verb~<Directory>~ sections or in .htaccess files. The only way to circumvent that is to rewrite to -.

\subsection{QSA - qsappend}
\label{qsaflag}
\index{QSA flag}
\index{Rewrite flag!QSA}

When the replacement URI contains a query string, the default behavior of RewriteRule is to discard the existing query string, and replace it with the newly generated one. Using the \verb~[QSA]~ flag causes the query strings to be combined.

Consider the following rule:

\begin{verbatim}
RewriteRule /pages/(.+) /page.php?page=$1 [QSA]
\end{verbatim}

With the \verb~[QSA]~ flag, a request for \verb~/pages/123?one=two~ will be mapped to \verb~/page.php?page=123&one=two~. Without the \verb~[QSA]~ flag, that same request will be mapped to \verb~/page.php?page=123~ - that is, the existing query string will be discarded.

\subsection{QSD - qsdiscard}
\label{qsdflag}
\index{QSD flag}
\index{Rewrite flag!QSD}


When the requested URI contains a query string, and the target URI does not, the default behavior of \verb~RewriteRule~ is to copy that query string to the target URI. Using the \verb~[QSD]~ flag causes the query string to be discarded.

This flag is available in version 2.4.0 and later.

Using \verb~[QSD]~ and \verb~[QSA]~ together will result in \verb~[QSD]~ taking precedence.

If the target URI has a query string, the default behavior will be observed - that is, the original query string will be discarded and replaced with the query string in the \verb~RewriteRule~ target URI.


\subsection{R - redirect}
\label{rflag}
\index{R flag}
\index{Rewrite flag!R}

Use of the \verb~[R]~ flag causes a HTTP redirect to be issued to the browser. If a fully-qualified URL is specified (that is, including \verb~http://servername/~) then a redirect will be issued to that location. Otherwise, the current protocol, servername, and port number will be used to generate the URL sent with the redirect.

Any valid HTTP response status code may be specified, using the syntax \verb~[R=305]~, with a 302 status code being used by default if none is specified. The status code specified need not necessarily be a redirect (3xx) status code. However, if a status code is outside the redirect range (300-399) then the substitution string is dropped entirely, and rewriting is stopped as if the L were used.

In addition to response status codes, you may also specify redirect status using their symbolic names: temp (default), permanent, or seeother.

You will almost always want to use \verb~[R]~ in conjunction with \verb~[L]~ (that is, use \verb~[R,L]~) because on its own, the \verb~[R]~ flag prepends \verb~http://thishost[:thisport]~ to the URI, but then passes this on to the next rule in the ruleset, which can often result in 'Invalid URI in request' warnings.

\subsection{S - skip}
\label{sflag}
\index{S flag}
\index{Rewrite flag!S}

The \verb~[S]~ flag is used to skip rules that you don't want to run. The syntax of the skip flag is \verb~[S=N]~, where N signifies the number of rules to skip (provided the RewriteRule and any preceding RewriteCond directives match). This can be thought of as a goto statement in your rewrite ruleset. In the following example, we only want to run the RewriteRule if the requested URI doesn't correspond with an actual file.

\begin{verbatim}
# Is the request for a non-existent file?
RewriteCond %{REQUEST_FILENAME} !-f
RewriteCond %{REQUEST_FILENAME} !-d
# If so, skip these two RewriteRules
RewriteRule .? - [S=2]

RewriteRule (.*\.gif) images.php?$1
RewriteRule (.*\.html) docs.php?$1
\end{verbatim}

This technique is useful because a \verb~RewriteCond~ only applies to the \verb~RewriteRule~ immediately following it. Thus, if you want to make a \verb~RewriteCond~ apply to several \verb~RewriteRule~s, one possible technique is to negate those conditions and add a \verb~RewriteRule~ with a \verb~[Skip]~ flag. You can use this to make pseudo if-then-else constructs: The last rule of the then-clause becomes skip=N, where N is the number of rules in the else-clause:

\begin{verbatim}
# Does the file exist?
RewriteCond %{REQUEST_FILENAME} !-f
RewriteCond %{REQUEST_FILENAME} !-d
# Create an if-then-else construct by skipping 3 lines if we meant to go to the "else" stanza.
RewriteRule .? - [S=3]

# IF the file exists, then:
    RewriteRule (.*\.gif) images.php?$1
    RewriteRule (.*\.html) docs.php?$1
    # Skip past the "else" stanza.
    RewriteRule .? - [S=1]
# ELSE...
    RewriteRule (.*) 404.php?file=$1
# END
\end{verbatim}

It is probably easier to accomplish this kind of configuration using the \verb~<If>~, \verb~<ElseIf>~, and \verb~<Else>~ directives instead. (See \ref{if}.)

\subsection{T - type}
\label{tflag}
\index{T flag}
\index{Rewrite flag!T}

Sets the MIME type with which the resulting response will be sent. This has the same effect as the \verb~AddType~ directive.

For example, you might use the following technique to serve Perl source code as plain text, if requested in a particular way:

\begin{verbatim}
# Serve .pl files as plain text
RewriteRule \.pl$ - [T=text/plain]
\end{verbatim}

Or, perhaps, if you have a camera that produces jpeg images without file extensions, you could force those images to be served with the correct MIME type by virtue of their file names:

\begin{verbatim}
# Files with 'IMG' in the name are jpg images.
RewriteRule IMG - [T=image/jpg]
\end{verbatim}

Please note that this is a trivial example, and could be better done using \verb~<FilesMatch>~ instead. Always consider the alternate solutions to a problem before resorting to rewrite, which will invariably be a less efficient solution than the alternatives.

If used in per-directory context, use only - (dash) as the substitution for the entire round of mod\_rewrite processing, otherwise the MIME-type set with this flag is lost due to an internal re-processing (including subsequent rounds of mod\_rewrite processing). The L flag can be useful in this context to end the current round of mod\_rewrite processing.

\section{Per-directory rewrites}

The rewrite engine may be used in .htaccess files and in <Directory> sections, with some additional complexity.
To enable the rewrite engine in this context, you need to set "RewriteEngine On" and "Options FollowSymLinks" must be enabled. If your administrator has disabled override of FollowSymLinks for a user's directory, then you cannot use the rewrite engine. This restriction is required for security reasons.

When using the rewrite engine in .htaccess files the per-directory prefix (which always is the same for a specific directory) is automatically removed for the RewriteRule pattern matching and automatically added after any relative (not starting with a slash or protocol name) substitution encounters the end of a rule set. See the RewriteBase directive for more information regarding what prefix will be added back to relative substitutions.

If you wish to match against the full URL-path in a per-directory (htaccess) RewriteRule, use the \verb~%{REQUEST_URI}~ variable in a RewriteCond.

The removed prefix always ends with a slash, meaning the matching occurs against a string which never has a leading slash. Therefore, a Pattern containing \verb~^/~ never matches in per-directory context.

Although rewrite rules are syntactically permitted in \verb~<Location>~ and \verb~<Files>~ sections, this should never be necessary and is unsupported.

\section{The Query String}

Many scenarios that come up on the support channels call for modifying a request based on the query string (the bit of a URL following a ?). This is not something \verb~RewriteRule~ can do, and requires the services of the \verb~RewriteCond~ directive. See Chapter \ref{rewritecond}.

\section{RewriteBase}

\chapter{RewriteCond}
\label{rewritecond}
\index{RewriteCond}

The \verb~RewriteCond~ directive attaches additional conditions on a \verb~RewriteRule~, and may also set backreferences that may be used in the rewrite target.



\chapter{RewriteMap}
\label{chapter_rewritemap}
\index{RewriteMap}

The \verb~RewriteMap~ directive gives you a way to call external mapping routines to simplify your \verb~RewriteRule~s. This external mapping can be a flat text file containing one-to-one mappings, or a database, or a script that produces mapping rules, or a variety of other similar things. In this chapter we'll discuss how to use a \verb~RewriteMap~ in a \verb~RewriteRule~ or \verb~RewriteCond~.

\section{Creating a RewriteMap}

The \verb~RewriteMap~ directive creates an alias which you can then invoke in either a \verb~RewriteRule~ or \verb~RewriteCond~ directive. You can think of it as defining a function that you can call later on.

The syntax of the \verb~RewriteMap~ directive is as follows:

\begin{verbatim}
RewriteMap MapName MapType:MapSource
\end{verbatim}

\textbf{MapName}: The name of the 'function' that you're creating

\textbf{MapType}: The type of the map. The various available map types are discussed below.

\textbf{MapSource}: The location from which the map definition will be obtained, such as a file, database query, or predefined function.

The \verb~RewriteMap~ directive must be used either in virtualhost context, or in global server context. This is because a \verb~RewriteMap~ is loaded at server startup time, rather than at request time, and, as such, cannot be specified in a \verb~.htaccess~ file.

\section{Using a RewriteMap}

Once you have defined a \verb~RewriteMap~, you can then use it in a \verb~RewriteRule~ or \verb~RewriteCond~ as follows:

\begin{verbatim}
RewriteMap examplemap txt:/path/to/file/map.txt
RewriteRule ^/ex/(.*) ${examplemap:$1}
\end{verbatim}

Note in this example that the \verb~RewriteMap~, named 'examplemap', is passed an argument, \verb~$1~, which is captured by the \verb~RewriteRule~ pattern. It can also be passed an argument of another known variable. For example, if you wanted to invoke the \verb~examplemap~ map on the entire requested URI, you could use the variable \verb~ %{REQUEST_URI}~ rather than \verb~$1~ in your invocation:

\begin{verbatim}
RewriteRule ^ ${examplemap:%{REQUEST_URI}}
\end{verbatim}

TODO: DEFAULT RESULT

\section{RewriteMap Types}

There are a number of different map types which may be used in a \verb~RewriteMap~.

\subsection{int}
\label{rewritemap_int}
\index{RewriteMap!int}

An \verb~int~ map type is an internal function, pre-defined by \verb~mod\_rewrite~ itself. There are four such functions:

\subsubsection{toupper}

The \verb~toupper~ internal function converts the provided argument text to all upper case characters.

\begin{verbatim}
# Convert any lower-case request to upper case and redirect
RewriteMap uc int:toupper
RewriteRule (.*?[a-z]+.*) ${uc:$1} [R=301]
\end{verbatim}

\subsubsection{tolower}

The \verb~tolower~ is the opposite of \verb~toupper~, converting any argument text to lower case characters.

\begin{verbatim}
# Convert any upper-case request to lower case and redirect
RewriteMap lc int:tolower
RewriteRule (.*?[A-Z]+.*) ${lc:$1} [R=301]
\end{verbatim}

\subsubsection{escape}
\subsubsection{unescape}

\subsection{txt}
\label{rewritemap_txt}
\index{RewriteMap!txt}

A \verb~txt~ map defines a one-to-one mapping from argument to target.

\subsection{rnd}
\label{rewritemap_rnd}
\index{RewriteMap!rnd}

A \verb~rnd~ map will randomly select one value from the specified text file.

\subsection{dbm}
\label{rewritemap_dbm}
\index{RewriteMap!dbm}

\subsection{prg}
\label{rewritemap_prg}
\index{RewriteMap!prg}

\subsection{dbd}
\label{rewritemap_dbd}
\index{RewriteMap!dbd}

